\documentclass{beamer}
\usepackage[utf8]{inputenc}
\usepackage[german]{babel}
\usepackage[T1]{fontenc}
\usepackage{amsmath}
\usepackage{amsfonts}
\usepackage{amssymb}
\usepackage{booktabs}
\usepackage{graphicx}
\usepackage{enumitem}
\usepackage{csquotes}
\usepackage{lmodern}
\usepackage{pifont}
\usepackage{hyperref}
\usepackage{xcolor}


%Navigationsleiste verschwinden lassen
\beamertemplatenavigationsymbolsempty

% \logo{\includegraphics[height= 0.5 cm]{Franz}}
\title[Soziale Ungleichheit in der Lebenserwartung]{\textbf{Soziale Ungleichheit der Lebenserwartung in Deutschland}}
\institute[]{Wissenschaftliches Arbeiten, TU Dortmund}
\author[Caroline Baer, Louisa Poggel]{Caroline Baer, Louisa Poggel}
\date{07. Dezember 2021}


\usetheme{CambridgeUS}
\usecolortheme{dolphin} %oder beaver oder crane (update 27.11, 13:29)
\useinnertheme{rounded}

\usepackage{tikz}
\definecolor{pbbg}{RGB}{90,90,90}% filling color for the progress bar
\definecolor{pbfill}{RGB}{90,90,90}% background color for the progress bar

\makeatletter
\def\progressbar@progressbar{} % the progress bar
\newcount\progressbar@tmpcounta% auxiliary counter
\newcount\progressbar@tmpcountb% auxiliary counter
\newdimen\progressbar@pbht %progressbar height
\newdimen\progressbar@pbwd %progressbar width
\newdimen\progressbar@tmpdim % auxiliary dimension
\progressbar@pbwd=\paperwidth
\progressbar@pbht=0.5ex
% the progress bar
\def\progressbar@progressbar{%
    \progressbar@tmpcounta= \insertframenumber % max = ?
    \progressbar@tmpcountb=\inserttotalframenumber      
    \progressbar@tmpdim=.5\progressbar@pbwd
    \multiply\progressbar@tmpdim by \progressbar@tmpcounta
    \divide\progressbar@tmpdim by \progressbar@tmpcountb
    \progressbar@tmpdim=2\progressbar@tmpdim
  \begin{tikzpicture}[rounded corners=1.5pt,very thin]
  \shade[top color=pbbg!20,bottom color=pbbg!20,middle color=pbbg!50]
       (0pt, 0pt) rectangle ++ (\progressbar@pbwd, \progressbar@pbht);

       \shade[draw=pbfill,top color=pbfill!50,bottom color=pbfill!50,middle color=pbfill] %
         (0pt, 0pt) rectangle ++ (\progressbar@tmpdim, \progressbar@pbht);
    \draw[color=normal text.fg!50]
      (0pt, 0pt) rectangle (\progressbar@pbwd, \progressbar@pbht)
        node[pos=0.5,color=normal text.fg] {\textnormal{%
        }%
    };
  \end{tikzpicture}%
}
\addtobeamertemplate{footline}  %---footline für unten (über lila Balken)------
%---------und headline für oben (über lila Balken)-----------------------------
{%
  \begin{beamercolorbox}[wd=\paperwidth,ht=2.0ex,center,dp=0ex]{white}%
    \progressbar@progressbar%
  \end{beamercolorbox}%
}
\makeatother  


\begin{document}

\begin{frame}
 \maketitle
\end{frame}

\begin{frame}
 \frametitle{Inhaltsverzeichnis}
  \tableofcontents  %[pausesections]
\end{frame}

\section{Motivation}

\begin{frame}{Veränderung der Lebenserwartung }
$\blacktriangleright$ \textbf{Erreichen des 60.Lebensjahres:}\\
 \vspace{0.4cm}
\hspace{0.5cm} \textbf{1880:}~$\frac{1}{3}$ der Bevölkerung       \hspace{0.5cm} $\Longrightarrow$ \hspace{0.5cm} \textbf{1975:} 75\% der Bevölkerung\\
 \vspace{0.6cm}
$\blacktriangleright$ \textbf{Vorausrechnungen 2050:} 30\% älter als 65 Jahre  
\vspace{0.2cm}
\pause
\begin{block}{Gründe:}
		\begin{itemize}
			\item[$\blacktriangleright$] Eindämmung der Infektionskrankheiten und Kindersterblichkeit
			\item[$\blacktriangleright$] Verringerung chronischer Krankheit im hohen Alter
			\item[$\blacktriangleright$] bessere Lebensbedingungen
		\end{itemize}
	\end{block}
\end{frame}



%\begin{frame}{Veränderung der Lebenserwartung}
%	\begin{block}{}
%		\begin{itemize}
%		\item[$\blacktriangleright$] 1880: nur ein Drittel der Bevölkerung erreicht das 60. Lebensjahr
%		\item[$\blacktriangleright$] 1975: sind es bereits 75\%
%%zu volle Folie	\item[$\blacktriangleright$] Ende 20.Jhd: nahe 90 \%
%		\item[$\blacktriangleright$] weiterer Anstieg erwartet
%		\end{itemize}
%	\end{block}
%	\pause
%	\begin{block}{}
%		\begin{itemize}
%			\item[$\blacktriangleright$] 2005: 19\% der Gesamtbevölkerung älter als 65
%			\item[$\blacktriangleright$] Vorausrechnung des Statistischen Bundesamtes\\ für 2050: 30\% älter als 65
%		\end{itemize}
%	\end{block}
%	\pause
%	\begin{block}{Gründe:}
%		\begin{itemize}
%			\item[$\blacktriangleright$] Eindämmung der Infektionskrankheiten und Kindersterblichkeit
%			\item[$\blacktriangleright$] Verringerung chronischer Krankheit im hohen Alter
%			\item[$\blacktriangleright$] bessere Lebensbedingungen
%		\end{itemize}
%	\end{block}
%\end{frame}

\begin{frame}
 \frametitle{Unterschiede in der Lebenserwartung}
\textbf{Differenz mittlere Lebenserwartung bei Geburt:}

niedrigste Einkommensgruppe \hfill höchste Einkommensgruppe

\hspace{1.5cm}\includegraphics[height=2cm]{Personen}\hspace{2cm} 
$\Longleftrightarrow$ \hspace{2cm}\includegraphics[height=2cm]{Personen}\\
\vspace{0.5cm}
\hspace{4.3cm} \textbf{Frauen: 4.4 Jahre}\\
\hspace{4.3cm} \textbf{Männer: 8.6 Jahre}
     
    
\end{frame}


\section{Hypothese}
\begin{frame}
 \frametitle{Hypothese: Lebenserwartung in Deutschland vom Einkommen stark beeinflusst}
 \textbf{Ungleichheit der Lebensbedingungen:}
 \vspace{0.5cm}
 \begin{itemize}
   \item [$\blacktriangleright$] Verteilung des Einkommens
   \item [$\blacktriangleright$] Bildungschancen
   \item [$\blacktriangleright$] Risiko chronischer Erkrankungen
   \item [$\blacktriangleright$] individuelles Gesundheitsverhalten
 \end{itemize}
 \vspace{0.5cm}
 $\rightarrow$ Verkürzte Lebenszeit sozial benachteiligter Bevölkerungsgruppen
 
\end{frame}


\section{Studie und Datenbasis}
\begin{frame}
\frametitle{Wichtige Datenquellen und Studien}
  \begin{block}{Sozio-oekonomische Panel (SOEP)}
   \begin{itemize}
     \item [$\blacktriangleright$] durch Deutsches Institut für   Wirtschaftsforschung (DIW)
     \item [$\blacktriangleright$] Panelstudie von 1992-2016
     \item [$\blacktriangleright$] Daten von  83.287 Personen (bezüglich obigen Zeitraumes)
 %   \item[$\blacktriangleright$] jährliche Haushaltsbefragung
     \item[$\blacktriangleright$] insgesamt 4.193 (dh. 5\%) Studienteilnehmer im beobachteten Zeitraum verstorben
   \end{itemize}
  \end{block}
  \begin{block}{Daten des Statistischem Bundesamt}
   \begin{itemize}
     \item [$\blacktriangleright$] Amtliche Periodensterbetafeln
     \item [$\blacktriangleright$] Statistik der natürlichen Bevölkerungsbewegung
   \end{itemize}
  \end{block}
\end{frame}

%--------------richtiger Ort?--------soll diese Folie drin bleiben?----------
%\begin{frame}
%\frametitle{Herausforderungen bei Datenerhebung und statistischer Analyse}
%	\begin{itemize}
%		\item [$\blacktriangleright$] keine amtliche Informationsquelle die 	Sterberegister mit sozialer Lage verknüpft
%		\item [$\blacktriangleright$] Austretende Studienteilnehmer (mit schlechter Gesundheit)
%			\begin{itemize}
%				\item $\rightarrow$ Unterschätzung Mortalität
%				\item $\rightarrow$ Überschätzung Lebenserwartung
%			\end{itemize}
%%		\item [$\blacktriangleright$]Theorie: Erhöhung der Lebenszeit in höchsten/mittleren Einkommensklassen stärker als in niedrigster Einkommensklasse
%%		\begin{itemize}
%%			\item $\rightarrow$ keine statistische Absicherung aufgrund zu niedriger\\
%%			\hspace{0.4cm} Fallzahlen (große Unsicherheit der Schätzer)
%%		\end{itemize}
%	\end{itemize}
%\end{frame}
%----------------------------------------> mündlich ergänzen


\section{Verwendete Methoden}
\begin{frame}{Netto-Äquivalenzeinkommen}
	\begin{itemize}
		\item[$\blacktriangleright$] Einkommen nach Berücksichtigung von Größe/Zusammensetzung des Haushaltes, sowie unterschiedlichen Einkommensbedarfes
\vspace{0.3cm}		
		\begin{itemize}
		\item[$\bullet$]	Addition des Einkommen des gesamten Haushalts \& Gewichtung nach neuer OECD-Skala
		%\item[$\bullet$] Gewichtung nach Alter und Anzahl der Personen gerichtet
\vspace{0.2cm}
		\item[$\Rightarrow$] Netto-Äquivalenzeinkommen $=\frac{\text{Summe der Nettoinkommen (in €)}}{\text{Summe der Personengewichte}}$
		\end{itemize}
\vspace{0.3cm}
		\item[$\blacktriangleright$] 1992-2016: mittlere Netto-Äquivalenzeinkommen (Median) $=$ 1.495€
	\end{itemize}
\end{frame}

\begin{frame}{Einkommensgruppen}
	\begin{block}{Einteilung in 5 Gruppen bzgl. des gesellschaftlichen Medians:}
		\begin{itemize}
			\item[$\blacktriangleright$] unter 60\%
			\item[$\blacktriangleright$] 60 bis unter 80\%
			\item[$\blacktriangleright$] 80 bis unter 100\%
			\item[$\blacktriangleright$] 100 bis unter 150\%
			\item[$\blacktriangleright$] über 150\%
		\end{itemize}
	\end{block}
	\begin{block}{Schwellenwerte:}
		\begin{itemize}
			\item[$\blacktriangleright$] 60\%:\, 897€ \\ $\rightarrow$ nach sozialpolitischer Definition von Armut betroffen oder gefährdet
			\item[$\blacktriangleright$] 150\%:\, 2.243€
		\end{itemize}
	\end{block}
\end{frame}



\section{Ergebnisse}
\begin{frame}
\textbf{Überlebensraten nach Geschlecht und Einkommen}
	\includegraphics[scale=0.18]{Abb.2_Überlebensraten nach Geschlecht und Einkommen_korrigiert}
	\emph{Quelle: SOEP, Periodensterbetafeln 1992-2016}
\end{frame}

\begin{frame}
\textbf{Mortalitätsrisiko vor einem Alter von 65 Jahren (\%) }
	\includegraphics[scale=0.17]{Grafik_Mortalitätsrisiko_neu}
	\emph{Quelle: SOEP, Periodensterbetafeln 1992-2016}
\end{frame}

\begin{frame}
  \textbf{Allgemeine und gesunde Lebenserwartung nach Einkommen und Geschlecht}
	\includegraphics[scale=0.15]{Tabelle_allgemeine_und_gesunde_Lebenserwartung_neu}
	\emph{Quelle: SOEP, Periodensterbetafeln 1995-2005}
\end{frame}
 
 
\section{Fazit}
\begin{frame}{Fazit - Lebenserwartung}
	\begin{block}{Veränderung der Lebenserwartung im Beobachtungszeitraum:}
		\begin{itemize}
			\item[$\blacktriangleright$] Frauen: 78,9 $\rightarrow$ 82,2 Jahre\\
			\item[$\blacktriangleright$] Männer: 72,3 $\rightarrow$ 77,4 Jahre
		\end{itemize}
	Zugewinn (in Jahren):
\hspace{0.45cm}	
	\begin{tabular}{lcc}
	\toprule
	 & \multicolumn{2}{c}{Einkommensgruppe}\\
	 & niedrigste  & höchste \\
	\midrule
	Frauen & 1,4  & 3,9  \\
	Männer & 4,2  & 6,9  \\
	\bottomrule	
	\end{tabular}
	\end{block}
%	\vspace{0.3cm}
%	\begin{block}{Veränderung der Lebenserwartung im Beobachtungszeitraum:}
%		\begin{itemize}
%			\item[$\blacktriangleright$] Frauen: 78,9 $\rightarrow$ 82,2 Jahre\\
%			\begin{itemize}
%			\item[$\bullet$] 1,4 Jahre Zugewinn (niedrigste)
%			\item[$\bullet$] 3,9 Jahre Zugewinn (höchste)
%			\end{itemize}								
%			\item[$\blacktriangleright$] Männer: 72,3 $\rightarrow$ 77,4 Jahre
%			\begin{itemize}
%				\item[$\bullet$] 4,2 Jahre Zugewinn (niedrigste Einkommensgr.)
%				\item[$\bullet$] 6,9 Jahre Zugewinn (höchste Einkommensgr.)
%			\end{itemize}
%		\end{itemize}
%	\end{block}
	\pause
	\begin{block}{Differenz zwischen niedrigster und höchster Einkommensgruppe}
		\begin{itemize}
			\item[$\blacktriangleright$] bzgl. mittlerer Lebenserwartung bei Geburt:\\ Frauen: 4,4 Jahre, Männer: 8,6 Jahre
			\item[$\blacktriangleright$] bzgl. fernerer Lebenserwartung ab einem Alter von 65 Jahren:\\ Frauen: 3,7 Jahre, Männer: 6,6 Jahre
		\end{itemize}
	\end{block}
\end{frame}

\begin{frame}{Fazit - Mortalität}
	\begin{block}{Verstorben vor Vollendung des 65. Lebensjahres:}	
\hspace{3cm}		
		\begin{tabular}{lcc}
			\toprule
			 & \multicolumn{2}{c}{Einkommensgruppe}\\
			 & niedrigste  & höchste \\
			\midrule
			Frauen & 13,2\%  & 8,2\% \\
			Männer & 27,2\%  & 13,6\%  \\
		\bottomrule	
		\end{tabular}
	\end{block}
%	\begin{block}{}
%		\begin{itemize}
%			\item[$\blacktriangleright$] 13,2\% der Frauen, 27,2\% der Männer aus der niedrigsten Einkommensgruppe sterben vor Vollendung des 65. Lebensjahres
%			\item[$\blacktriangleright$] 8,2\% der Frauen, 13,6\% der Männer aus der höchsten Einkommensgruppe sterben vor Vollendung des 65. Lebensjahres
%		\end{itemize}
%	\end{block}
	\pause
	\begin{block}{Mortalitätsrisiko in der niedrigsten Einkommensgruppe}
		\begin{itemize}
			\item[$\blacktriangleright$] bis zum Alter von 50 Jahren:
			\begin{itemize}
				\item[$\bullet$] Frauen: 2,2-fach höher, Männer: 2,4-fach höher
			\end{itemize}
			\item[$\blacktriangleright$] ab einem Alter von 51 Jahren:
			\begin{itemize}
				\item[$\bullet$] Frauen: 1,5-fach höher, Männer: 1,9-fach höher 
			\end{itemize}
		\end{itemize}
	\end{block}
\end{frame}


\section{Quellen}
\begin{frame}
\frametitle{Quellen}
   \begin{itemize}
    \item \textbf{T.Lampert, J.Hoebel, et.al (2019) Journal of Health Monitoring}\\
    Abschnitt: Soziale Unterschiede in der Mortalität und Lebenserwartung in Deutschland - Aktuelle Situation und Trends (S.~3-15)
    
    \item \textbf{T.Lampert, L.E.Kroll, et.al (2007) Aus Politik und Zeitgeschichte - Gesundheit und soziale Ungleichheit}\\
Abschnitt: Soziale Ungleichheit der Lebenserwartung in Deutschland (S. 11-18)

%    \item \textbf{Symbolbild Personen:}
%    https://icon-icons.com/de/symbol/Benutzer-Gruppe-Personen-Kunden-Klienten/72448
    
    
  \end{itemize}

\end{frame}


\section{Diskussion}
\begin{frame}{Diskussionsfragen}
	\begin{block}{Frage 1:}
		 Was glaubt ihr wie sich die Lebenserwartung in den nächsten Jahren entwickeln wird?
	\end{block}
	\pause  %zweite Frage erst nach Diskussion erster Frage einblenden
	\begin{block}{Frage 2:}
		 Habt ihr Vorschläge wie man die soziale Ungleichheit in der Lebenserwartung verringern oder gar aufheben könnte?
	\end{block}
\end{frame}



\end{document}